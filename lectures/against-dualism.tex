\section{Againt Dualism}
\label{against-dualism}

Notes from this section are roughly based on the chapter "The Flight from Dualism" in \cite{pomac}.

\subsection{Introduction}

There seems to be a clear distinction between things that are material and mental.

In the material world we have things like
\begin{itemize}
    \item A person's weight
    \item A person's hair color
    \item A person's genetic makeup
\end{itemize}
whereas in the mental world we have things like
\begin{itemize}
    \item How a person is feeling
    \item What a person is thinking
    \item A person's intellect
\end{itemize}

There are two ways to explain these differences. The first is \vocab{dualism}, which posits that the material and mental worlds are fundamentally separated.

On the other hand, we have \vocab{materialism} and \vocab{idealism}, which are both a form of monism.

Materialism argues that everything can be simplified to the material: that our mental states can be explained from examining our material state. Idealism is the opposite: it says that the material is explained purely in terms of the mental.

Nowadays, materialism is orthodoxy and both dualism and idealism are discredited. Idealism mainly because it gets compared to \vocab{vitalism} -- the idea that the living and inanimate are fundamentally different. That is, perhaps the living possesses some kind of vital force that the inanimate does not. This is often discredited nowadays ever since the rise of modern chemistry.

