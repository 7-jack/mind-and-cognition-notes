\section{Againt Dualism}
\label{against-dualism}

Notes from this section are roughly based on the chapter "The Flight from Dualism" in \cite{pomac}.

\subsection{Introduction}

There seems to be a clear distinction between things that are material and mental.

In the material world we have things like
\begin{itemize}
    \item A person's weight
    \item A person's hair color
    \item A person's genetic makeup
\end{itemize}
whereas in the mental world we have things like
\begin{itemize}
    \item How a person is feeling
    \item What a person is thinking
    \item A person's intellect
\end{itemize}

There are two ways to explain these differences. The first is \vocab{dualism}, which posits that the material and mental worlds are fundamentally separated.

On the other hand, we have \vocab{materialism} and \vocab{idealism}, which are both a form of monism.

Materialism argues that everything can be simplified to the material: that our mental states can be explained from examining our material state. Idealism is the opposite: it says that the material is explained purely in terms of the mental.

Nowadays, materialism is orthodoxy and both dualism and idealism are discredited. Idealism mainly because it gets compared to \vocab{vitalism} -- the idea that the living and inanimate are fundamentally different. That is, perhaps the living possesses some kind of vital force that the inanimate does not. This is often discredited nowadays ever since the rise of modern chemistry.

\subsection{For Dualism}

Some sketches of arguments for dualism and some sketches of arguments against it.

\begin{definition}[Leibniz's Law]
    Also known as \vocab{the indiscerniblity of identicals}, it states that if $$x = y$$ then all properties of $x$ are a property of $y$, and vice versa.
\end{definition}

With Leibniz's law, it is quite easy to see a sketch of an argument for dualism: simply observe that there seem to be qualities that physical states have that our mental states do not. For example, physical states have \textit{mass}. Do the state of pain have mass? If not, then clearly the mental and physical (material) cannot be equal, and so the argument for dualism goes.

The empirical response is that things that "seem true" are not necessarily true.
\begin{remark}
    Here in the textbook, it seems the authors make the claim that light has mass. It must be noted that when claims like this are made, it is about "relativistic mass."
\end{remark}

The argument above was applied to mental and physical \textit{states}. It can also be applied to \textit{beliefs}. Take, for example, the belief that snow is white. It lacks location and temperature which all physical states have.

When talking about beliefs, it is important to make the distinction  between the \textit{state} of belief, and what is believed itself (i.e. the \textit{contents} of the belief). If we talk about the state of the belief, the conversation reduces to a conversation about mental and physical states. If it is about the content of the belief, this is much more complicated and materialists have some things to say about this (to be covered later).

